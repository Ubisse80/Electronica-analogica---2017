\documentclass[11pt,a4paper,titlepage]{article}
\usepackage[utf8]{inputenc}
\usepackage[portuguese]{babel}
\usepackage[T1]{fontenc}
\usepackage{fancyhdr}
\usepackage{amsmath}
\usepackage{amsfonts}
\usepackage{amssymb}
\usepackage{makeidx}
\usepackage{graphicx}
\usepackage{lmodern}
\usepackage{wrapfig}
\usepackage{color}
\usepackage[table]{xcolor}
\usepackage{physics}
\usepackage{float}
\usepackage{fourier}
\usepackage[left=1.5cm,right=1.5cm,top=1cm,bottom=2cm]{geometry}
\author{Bartolomeu J. Ubisse}
\pagestyle{fancy}
\renewcommand{\headrulewidth}{0pt}
\renewcommand{\footrulewidth}{1pt}
\fancyfoot[L]{  UEM/FC/DF - 2022}
\fancyfoot[c]{}
\fancyfoot[r]{\thepage}
\usepackage{fancyhdr, lastpage}
%\pagestyle{fancy}
\fancyfoot[r]{{P\'ag. \thepage} / \pageref{LastPage}}

\begin{document}
\begin{figure}[htb]

\centering
\includegraphics[scale=1]{UEM-logotipo}
\end{figure}
\centering
{ \Large Universidade Eduardo Mondlane}\\[0.3cm] 
\large Faculdade de Ci\^encias\\[0.2cm]
 \large Departamento de F\'isica\\[0.5cm]



\begin{flushleft}
\tt Teste II - E. Anal\'ogica\hspace{0.25cm} |Data:$12/07/2022$\hspace{0.25cm}|Hora:$13:00-15:30$ hrs\vspace{0.08cm}
\hrule
\end{flushleft}

 \begin{table}[htb]
\centering
\begin{tabular}{p{16.55cm}}
\hline
\cellcolor[gray]{0.95}
Responda atentamente as perguntas que lhe s\~ao colocadas e mostre todos os passos necess\'arios.\\
 \hline
\end{tabular}
\end{table}

\begin{enumerate}

\begin{minipage}[c]{12cm}
\item Determine o equivalente Norton da Fig.\ref{f2} e calcule a queda de tensão na resistência de carga (RL)

 \item Explique o que entende por: \textbf{a)} Impureza doadora; \textbf{b)} Junção PN; \textbf{c)} Recombinação; \textbf{d)} Camada de depleção .

\item Apresente de uma forma sequenciada todas as etapas de um processo de rectificação de um sinal sinusoidal e na saida de cada etapa, esboce a forma do sinal correspondente.

\end{minipage}\hfill
\begin{minipage}[c]{5cm}
 \begin{figure}[H]
   \centering
   \includegraphics[scale=0.5]{Norton}
   \caption{}
   \label{f2}
   \end{figure}
\end{minipage}

\begin{minipage}[c]{12cm}
\item Determine a forma de onda na sa\'ida do circuíto da Fig.\ref{f3} sabendo que o sinal sinusoidal de entrada tem um pico de 6V (V$_p$ = 6V).
\item O transístor bipolar de junção (TBJ) pode trabalhar em três regiões. Diga quais são essas regiões e qual é a utilidade do TBJ para cada uma delas.
\end{minipage}\hfill
\begin{minipage}[c]{5cm}
 \begin{figure}[H]
   \centering
   \includegraphics[scale=0.6]{limitador}
   \caption{}
   \label{f3}
   \end{figure}
\end{minipage}


\begin{minipage}[c]{12cm}
\item Determine I$_B$, I$_C$ e I$_E$ do circuito da Fig.\ref{f5}. Explique para que servem os capacitores C$_1$, C$_2$ e C$_3$.
\item Explique como \'e que funciona o multivibrador ast\'avel representado pela Fig.\ref{flip}

 \begin{figure}[H]
   \centering
   \includegraphics[scale=0.4]{flip_flop}
   \caption{}
   \label{flip}
   \end{figure}
\end{minipage}\hfill
\begin{minipage}[c]{5cm}
 \begin{figure}[H]
   \centering
   \includegraphics[scale=0.7]{TBJ}
   \caption{}
   \label{f5}
   \end{figure}

\end{minipage}

\end{enumerate}
\vspace{1.0cm}

\Huge \textbf{Bom Trabalho!}

\end{document}