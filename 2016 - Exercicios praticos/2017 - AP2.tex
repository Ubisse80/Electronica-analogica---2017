\documentclass[12pt,a4paper,titlepage]{report}
\usepackage[utf8]{inputenc}
\usepackage[portuguese]{babel}
\usepackage[T1]{fontenc}
\usepackage{fancyhdr}
\usepackage{amsmath}
\usepackage{amsfonts}
\usepackage{amssymb}
\usepackage{makeidx}
\usepackage{graphicx}
\usepackage{lmodern}
\usepackage{wrapfig}
\usepackage{color}
\usepackage{float}
%\usepackage{fourier}
\usepackage[left=2cm,right=2cm,top=1cm,bottom=2cm]{geometry}
\author{Bartolomeu J. Ubisse}
\pagestyle{fancy}
\renewcommand{\headrulewidth}{0pt}
\renewcommand{\footrulewidth}{1pt}
\fancyfoot[L]{\tt Dr.A.Macamo; B.Ubisse \& H. Marrenjo | UEM - 2017}
\fancyfoot[c]{}
\fancyfoot[r]{\thepage}
\begin{document}


\begin{figure}[htb]

\centering
\includegraphics[scale=1]{UEM-logotipo}
\end{figure}
\centering
{ \Large Universidade Eduardo Mondlane}\\[0.3cm] 
\large Faculdade de Ci\^encias\\[0.2cm]
 \large Departamento de F\'isica\\[0.5cm]

\textsc{Electr\^onica B\'asica} \\[1cm]
\begin{flushleft}
\tt 2017-AP\# 2-Bandas Energ\'eticas e Materiais Semicondutores\\
\hrulefill
\end{flushleft}

\begin{enumerate}
\item Explique o que entende por uma banda energ\'etica.
\item Como \'e que se explica consider-se banda energ\'etica como cont\'inua enquanto que os n\'iveis energ\'eticos num dado \'atomo s\~ao discretos. 
\item Supondo que uma firma dedicada a fabrico de materiais el\'ectricos  apresenta resultados de an\'alise de uma amostra cuja  energia de banda proibida seja $0<E_g<2.5 eV$ para a sua identifica\c c\~ao. Explique com detalhes o que \'e esse material no que concerne \`a condutibilidade el\'ectrica.
\item Explique o que entende por banda proibida.
\item Dada a fig.\ref{f1}, determine as magnitudes da energia proibida destes materiais a 300K e a 400K.\\
 i) Ser\'a que obedecem a seguinte rela\c c\~ao $E_G(T)=E_G(0)-\frac{\alpha T^2}{T+\beta}$ ($\alpha$ e $\beta$ s\~ao par\'ametros de ajustes e os seus valores podem se ver na fig.\ref{f1}) ?\\
 ii) Ser\'a que podemos optar em elevar a temperatura de um semicondutor para  que tenhamos maior popula\c c\~ao de electr\~oes na banda de condu\c c\~ao?

\item Determine a energia t\'ermica do electr\~ao a uma temperatura de 300K. Como \'e que acha que o electr\~ao ganha essa energia? 
\item O que entende por liga\c c\~ao covalente?

\item Explique o que entende por dopagem e qual \'e a sua finalidade.

\item Explique em que difere um s/cond. intr\'inseco do extr\'inseco.
\item Explique o que entende por impurezas doadora e aceitadora. Fundamente a sua resposta com exemplos ilucidativos.
\item Quais s\~ao os portadores maiorit\'arios e minorit\'arios num material do tipo N?
\item Que tipo de i\~oes existem num s/condutor do tipo P?


\begin{figure}[h]
\centering
\includegraphics[scale=0.8]{Egf(T)}
\caption{\footnotesize Banda proibida em fun\c c\~ao da temperatura}
\label{f1}
\end{figure}

\end{enumerate}







\end{document}